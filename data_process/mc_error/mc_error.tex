\documentclass{article}
%=============================================================================80
%	                          Packages                                     %
%==============================================================================%
\usepackage[utf8]{inputenc}
\usepackage{graphicx}
\usepackage{amsmath}
\usepackage{amssymb}
\usepackage{braket}
\usepackage{float}
\usepackage{subcaption}
\usepackage[margin=0.7in]{geometry}
\usepackage[version=4]{mhchem}
\usepackage{hyperref}
%==============================================================================%
%                           User-Defined Commands                              %
%==============================================================================%
\newcommand{\be}{\begin{equation}}
\newcommand{\ee}{\end{equation}}
\newcommand{\benum}{\begin{enumerate}}
\newcommand{\eenum}{\end{enumerate}}
\newcommand{\pd}{\partial}
%==============================================================================%
%                             Title Information                                %
%==============================================================================%
\title{Chapter 1; Preliminaries}
\author{Shane Flynn}
%==============================================================================%
\begin{document}

\section*{Monte Carlo Error Analysis}
This code computes a single vertical error bar associated with a column of data generated as a function of iteration. 
The user runs a total of N$_{MC}$ steps in their Monte Carlo Simulation, and generates an associated data point at each step. 
N$_{MC}$ most likely is a very large number, therefore during the simulation the user computes a partial average at every i$^{th}$ step, therefore the simulation outputs a data vector running from 1 to N.
\be
N = \frac{N_{MC}}{i}
\ee
Therefore the user will be expected to provide the size of their data to be processed (N) ahead of time.

In a Monte-Carlo simulation the initial data may be noisy (you have a small number of points) it is therefore common to actually ignore the first few data points (even if already partial averaged). 
The user should provide an integer, skip, that sets the number of data points to ignore for the analysis. 
Therefore the data to process runs from skip to N.

\subsection*{Second Partial Average}
If there is a lot of data, the user may want to compute another partial average (a partial average of the partially averaged data) just to make the data more manageable (we have written a code for each case).
In this case the user should provide the frequency with which to partial average the data (an integer defining how many terms to use in the second partial average). 

A programming complication, there is no guarantee that the user will evenly divide the data intervals, in the code we compute a parameter (adjust) to shift the data by to make sure we have a whole number for the interval (we throw out more data to ensure this). 
\be
\text{adjust} = mod[(N-skip), \text{p freq}]
\ee
So adjust provides the remainder of this division, and we will now process data starting at skip+adjust. 

For computing the standard deviation we will need the avarage of this data
\be
\text{average} = \frac{1}{N - (\text{skip}+\text{adjust})} \sum_{i=(\text{skip}+\text{adjust})}^{N} \text{data}(i)
\ee

The code then computes the partial averages over this subset of the data producing 
\be
\text{size(p data)} = \frac{N - (\text{skip} + \text{adjust})}{\text{p freq}}
\ee

We then compute the variance over this partially averaged data relative to the average. 
\be
\sigma^2 = \frac{\sum_{k=1}^{\text{size(p data)}} \left(\text{p data(k)} - \text{average}\right)^2}{(\text{size(p data - 1)})(\text{size p data})}
\ee
Where our denominator is adjusted due to the finite populaiton of the ismulation (the minus 1 term). 

And finally the standard deviation is simply the square root
\be
\sigma = \sqrt{\sigma^2}
\ee

\subsection*{Single Partial Average}
If the user provides a set of data from 1 to N, we can also compute the error directly from this data.
Again the user provides a skip parameter and the average is computed over the used data. 
\be
\text{average} = \frac{1}{N-\text{skip}} \sum_{i=\text{skip}}^N data(i)
\ee
And the variance is directly computed
\be
\sigma^2 = \frac{\sum_{k=\text{skip}}^{N} \left(\text{data(k)} - \text{average}\right)^2}{(N-\text{skip}-1)(N-\text{skip})}
\ee
And finally the standard deviation is simply the square root
\be
\sigma = \sqrt{\sigma^2}
\ee







\end{document}
